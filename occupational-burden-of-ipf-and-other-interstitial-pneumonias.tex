\documentclass[a4paper,12pt]{article}

%set font to Arial
%\usepackage{fontspec}
%\setmainfont{Arial}
\usepackage{helvet}
\renewcommand{\familydefault}{\sfdefault}

%graphics
\usepackage{graphicx}
\usepackage{pslatex}
\usepackage{pstricks}

%math equations
\usepackage{amsmath}

%display URLS
\usepackage{url}

%hyperlinks
\usepackage{hyperref}

%comments
\usepackage{verbatim} 

%nice tables
\usepackage{booktabs}
\newcommand{\ra}[1]{\renewcommand{\arraystretch}{#1}}

%nice references
\usepackage[super]{natbib}

%some maths
\usepackage{amsmath}

\newpage\title{\bf Occupational burden of IPF and other interstitial pneumonias)} 
\author{Carl Reynolds \\
\small National Heart \& Lung Institute, Imperial College London }

\pagenumbering{gobble}

\begin{document}

%what does a task force do?
%what is the purpose of our task force? for me is to summarise the occupational burden of IPF and other interstitial pneumonias


\section*{\centering Occupational Burden of IPF and other interstitial pneumonias}

Idiopathic pulmonary fibrosis (IPF) is a diagnosis of exclusion. It is made in the
presence of a usual interstitial pneumonitis (UIP) pattern on high resolution CT scan
or biopsy. The diagnosis requires that known causes of interstitial lung disease
(such as drug toxicity, connective tissue disease, domestic, and occupational or
environmental exposures) be excluded\cite{Travis2013}.



\section{Search strategy}

We searched pubmed and google scholar for "idiopathic pulmonary fibrosis" and synonyms including "cryptogenic fibrosing alveolitis" and "usual interstitial pneumonia" in combination with the term "occupation" and synonyms. Where we found relevant studies or review articles we also looked at the papers these papers cited and the papers that cited them. 

\section{Results}

We identified six previous review articles (Table ~\ref{table:reviews})\cite{Turner-Warwick1998} and 12 relevant articles and one abstract (Table ~\ref{table:papers}).

There are many review articles of the epidemiology of interstitial lung disease that do not necessarily focus on IPF and only briefly mention occupational factors. We selected review articles that specifically deal with occupational factors in IPF and cite the case-control studies identified.


%?plausibility
%epi and experimental evidence

%reviews of occupational ILD cover IPF
%
%something on difficulty looking at causes of rare disease
%+moving targets nomenclature wise
%?ripped from chapter 25?
%need to decide if include additional sources e.g glazer also published review in clin pulm med and an ers chapter (neither indexed by pubmed)
\cite{Turner-Warwick1998} Discusses difficulty in attribution/causality


\begin{table}
    \begin{tabular}{lp{6cm}ll}
    \textbf{year} & \textbf{title} &       \textbf{first author} &            \textbf{journal} \\
    \midrule
    1998    &  In search of a cause of cryptogenic fibrosing alveolitis (CFA): one initiating factor or many?       &  Turner-Warwick M &              Thorax \\
    2001    &               Occupational dust exposure and the aetiology of cryptogenic fibrosing alveolitis. &         Hubbard R &  Eur Respir J Suppl \\
    2005    &  Is Idiopathic Pulmonary Fibrosis an Environmental Disease?
            &          Taskar V &  Proc Am Thorac Soc \\
    2015    &                            Asbestosis and environmental causes of usual interstitial pneumonia.  &          Gulati M &  Curr Opin Pulm Med \\
    \bottomrule
    \end{tabular}
    \caption{Previous review articles of IPF and occupational factors}
    \label{table:reviews}
\end{table}

\begin{table}
    \begin{tabular}{lp{6cm}ll}
    \textbf{year} & \textbf{title} &      \textbf{first author} &        \textbf{journal} \\
    \midrule
    1990    &   What causes cryptogenic fibrosing alveolitis? A case-control study of environmental exposure to dust. &                    Scott J &                         BMJ \\
    1994    &                                                        Idiopathic pulmonary fibrosis. Epidemiologic approaches to occupational exposure. &                     Iwai K &   Am J Respir Crit Care Med \\
    1996    &                                           Occupational exposure to metal or wood dust and aetiology of cryptogenic fibrosing alveolitis. &                  Hubbard R &                      Lancet \\
    1998    &                                                         Case-control study of idiopathic pulmonary fibrosis and environmental exposures. &                   Mullen J &         J Occup Environ Med \\
    2000    &                                                                               Risk of cryptogenic fibrosing alveolitis in metal workers. &                  Hubbard R &                      Lancet \\
    2000    &  Occupational and environmental risk factors for idiopathic pulmonary fibrosis: a multicenter case-control study. Collaborating Centers. &             Baumgartner KB &              Am J Epidemiol \\
    2005    &                                                       Occupational and environmental factors and idiopathic pulmonary fibrosis in Japan. &                   Miyake Y &               Ann Occup Hyg \\
    2007    &                                                                                     Occupational exposure and severe pulmonary fibrosis. &                Gustafson T &                  Respir Med \\
    2008    &                                                     Occupational risks for idiopathic pulmonary fibrosis mortality in the United States. &                Pinheiro GA &  Int J Occup Environ Health \\
    2010    &                                            Risk factors for idiopathic pulmonary fibrosis in a Mexican population. A case-control study. &  García-Sancho FMC &                  Respir Med \\
    2012    &                Occupational and environmental risk factors for idiopathic pulmonary fibrosis in Egypt: a multicenter case-control study. &                Awadalla NJ &     Int J Occup Environ Med \\
        2013    &                                           Risk factors for idiopathic pulmonary fibrosis in Southern Europe: A case-control study. (abstract only) &                  Paolocci G &      ERS  \\
    2014    &    Effects of smoking, gender and occupational exposure on the risk of severe pulmonary fibrosis: a population-based case-control study. &                  Ekström M &                    BMJ Open \\
    \bottomrule
    \end{tabular}
    \caption{Previous IPF case-control studies looking at occupational exposure}
    \label{table:papers}
\end{table}

\section{Meta-analysis}
feqfe

\clearpage

%%%%%%%%%%%%%%%%%%%%%%%%%%
\makeatletter
 \def\@biblabel#1{#1}
\makeatother
%%%%%%%% gets rid of bracket around numbers in bibliography
%%%%%%%%%%%%%%%%%%%%%%%%%%%

\bibliographystyle{unsrtnat}
\bibliography{ipfjes}


\end{document}
