\documentclass[a4paper,12pt]{article}

%set font to Arial
%\usepackage{fontspec}
%\setmainfont{Arial}
\usepackage{helvet}
\renewcommand{\familydefault}{\sfdefault}

%graphics
\usepackage{graphicx}
\usepackage{pslatex}
\usepackage{pstricks}

%math equations
\usepackage{amsmath}

%display URLS
\usepackage{url}

%hyperlinks
\usepackage{hyperref}

%comments
\usepackage{verbatim} 

%nice tables
\usepackage{booktabs}
\newcommand{\ra}[1]{\renewcommand{\arraystretch}{#1}}

%nice references
\usepackage[super]{natbib}

%some maths
\usepackage{amsmath}

\newpage\title{\bf Occupational burden of IPF and other interstitial pneumonias)} 
\author{Carl Reynolds \\
\small National Heart \& Lung Institute, Imperial College London }

\pagenumbering{gobble}

\begin{document}

%what does a task force do?

%what is the purpose of our task force? for me is to summarise the occupational burden of IPF and other interstitial pneumonias


\section*{\centering Occupational Burden of IPF and other interstitial pneumonias}

Idiopathic pulmonary fibrosis (IPF) is a diagnosis of exclusion. It is made in the
presence of a usual interstitial pneumonitis (UIP) pattern on high resolution CT scan
or biopsy. The diagnosis requires that known causes of interstitial lung disease
(such as drug toxicity, connective tissue disease, domestic, and occupational or
environmental exposures) be excluded\cite{Travis2013}.



\section{Search strategy}

We searched pubmed and google scholar for "idiopathic pulmonary fibrosis" and synonyms including "cryptogenic fibrosing alveolitis" and "usual interstitial pneumonia" in combination with the term "occupation" and its synonyms. Where we found relevant studies or review articles we also looked at the papers these papers cited and the papers that cited them. 

\section{Results}

\subsection{IPF}
We identified four review articles covering occupational exposures in IPF (Table ~\ref{table:reviews})\cite{Turner-Warwick1998} and 14 relevant case-control studies (Table ~\ref{table:papers}).

There are many review articles of the epidemiology of interstitial lung disease that do not necessarily focus on IPF and only briefly mention occupational factors. We selected review articles that specifically deal with occupational factors in IPF and cite the case-control studies and case-reports identified.

Turner-Warick (1998)\cite{Turner-Warwick1998} discusses potential difficulties in establishing attribution and causality in IPF. She observes that there is variation in clinical practice with respect to the standard applied to exclude IPF; some clinicians exclude IPF when exposure to a potential cause is identified, others only when there is clear exposure to an established cause. 

She explains that diagnosis based on radiologic and clinical findings, and not on lung biopsy or bronchioalveolar lavage, may result in initiating agents for disease being overlooked. Further, that exposures such as asbestos, silica, coal, graphite, hard metal, and avian proteins, may result in disease that can not be differentiated from IPF.    

Reviewing the epidemiology of IPF and case-control studies to date Hubbard (2001)\cite{Hubbard2001} describes the association of IPF with occupational exposures to metal and wood and estimates that 10\% of IPF cases may be due to occupatioanl metal exposure and 5\% of cases to wood.    

Taskar and Coultas (2005)\cite{Taskar2006} reviewed and performed a meta-analysis of six case-control studies investigating environmental and occupational exposures in IPF. They reported population attributable risk percentages for agriculture and farming (20.8\%), livestock (4.1\%), wood dust (5\%), metal dust (3.4\%), stone/sand/silica (3.5\%), and smoking (49.1\%).

Gulati and Redlich's (2015)\cite{Gulati2015} review of exposures causing usual interstitial pneumonia highlights that asbestosis may appear indistinguishable from IPF and summarises previous case-control studies but did not pool studies to perform a meta-analysis. We address this, considering all case-control studies to date (Table ~\ref{table:papers}), below.  

\subsection{Other interstitial pneumonias}

\subsubsection{Pulmonary alveolar proteinosis}

Case series provide evidence of the potential contribution of occupational exposures to PAP.  Estimates of exposure prevalence among cases range from 0\% to >50\%, depending on the series and definition used (refs).  One single-institution retrospective study of 23 cases in the United States reported none of the patients had a history of “unusual occupational exposure” \cite{Kariman1984}.  Conversely, a similar series of 70 patients in Germany found 54\% had past occupational dust or fume
exposure, most commonly to silica, aluminum, and sawdust \cite{Bonella2011}, and a multi-center series of 41 patients from France reported a prevalence of occupational dust exposure of 39\% \cite{Briens2002}.  The largest series to date, with 223 cases in Japan, found that 32\% of men and 13\% of women reported inhalational dust exposure \cite{Inoue2008}.  

Several case-control studies have addressed the role of occupational exposures in PAP.  McEuen and colleagues \cite{McEuen1978} examined lung tissue of 37 cases and 13 controls, documenting significantly higher birefringent particle counts in cases than controls.  In addition, occupational exposures were ascertained in 13 (35\%) cases.  A similar study of 24 cases and 5 controls found more birefringent particles in 78\% of cases and increased inorganic particle concentration
in all cases \cite{Abraham1986}.  More recently, 34.2\% (13/38) of cases at a referral center in China were found to have occupational inhalational exposure, compared to 19.6\% (19/97) of hospital controls, although this difference was not statistically significant (P=0.072) \cite{Xiao2015}.   

Most idiopathic PAP is now recognized as autoimmune, involving autoantibodies to granulocyte-macrophage colony stimulating factor (GM-CSF) (ref).  Although cases related to occupational exposures traditionally have been categorized separately, some evidence suggests occupational exposures contribute to autoimmune PAP (ref).  A case report of PAP and silicosis in a former coal miner noted he had elevated serum anti-GM-CSF autoantibodies, whereas 50 cases of silicosis without PAP did
not \cite{Hosokawa2004}.  Another case of PAP with autoantibodies to GM-CSF was reported in an indium-tin oxide (ITO) production worker \cite{Cummings2010}.  However, subsequent testing of 17 ITO workers with lung disease in Japan did not detect these antibodies (ref).  The Japanese case series and the Chinese case-control study reported occupational inhalational exposure in about a third of those with autoimmune PAP \cite{Inoue2008}; \cite{Xiao2015}).  In the German case-control study, history of
occupational dust or fume exposure was associated with GM-CSF autoimmunity \cite{Bonella2011}.  

\subsubsection{Organizing Pneumonia}
%chap 25

Case reports of OP in patients with occupational exposures include cases in a spice processing worker, a floor cleaner exposed to benzalkonium compounds, a chemical factory worker with massive exposure to acetic acid, a worker exposed to hydrogen sulfide, a laboratory worker exposed to ortho-phenylenediamine, an art restorer exposed to gold dust, and a painter exposed to titanium nanoparticles \cite{Alleman2002}\cite{Stefano2003}\cite{Sheu2008}\cite{Doujaiji2010}\cite{Sanchez-Ortiz2011}\cite{Ribeiro2011}\cite{Cheng2012}.

An outbreak of OP occurred in the Spanish textile industry in the 1990s.  Case series documented 
Carl to add from Parkes book and not forgetting our lit search summary doc.

%other interstitial pneumonia section?

%is exluding Lee2015 reasonable?

%?plausibility
%epi and experimental evidence

%reviews of occupational ILD cover IPF
%
%something on difficulty looking at causes of rare disease
%+moving targets nomenclature wise
%?ripped from chapter 25?
%need to decide if include additional sources e.g glazer also published review in clin pulm med and an ers chapter (neither indexed by pubmed)

\begin{table}
    \begin{tabular}{lp{6cm}ll}
    \textbf{year} & \textbf{title} &       \textbf{first author} &            \textbf{journal} \\
    \midrule
    1998    &  In search of a cause of cryptogenic fibrosing alveolitis (CFA): one initiating factor or many?       &  Turner-Warwick M &              Thorax \\
    2001    &               Occupational dust exposure and the aetiology of cryptogenic fibrosing alveolitis. &         Hubbard R &  Eur Respir J Suppl \\
    2005    &  Is Idiopathic Pulmonary Fibrosis an Environmental Disease?
            &          Taskar V &  Proc Am Thorac Soc \\
    2015    &                            Asbestosis and environmental causes of usual interstitial pneumonia.  &          Gulati M &  Curr Opin Pulm Med \\
    \bottomrule
    \end{tabular}
    \caption{Previous review articles of IPF and occupational factors}
    \label{table:reviews}
\end{table}

\begin{table}
    \begin{tabular}{lp{6cm}ll}
    \textbf{year} & \textbf{title} &      \textbf{first author} &        \textbf{journal} \\
    \midrule
            1990    &                                    What causes cryptogenic fibrosing alveolitis? A case-control study of environmental exposure to dust. &                    Scott J &                         BMJ \\
            1994    &                                                        Idiopathic pulmonary fibrosis. Epidemiologic approaches to occupational exposure. &                     Iwai K &   Am J Respir Crit Care Med \\
            1996    &                                           Occupational exposure to metal or wood dust and aetiology of cryptogenic fibrosing alveolitis. &                  Hubbard R &                      Lancet \\
            1998    &                                                         Case-control study of idiopathic pulmonary fibrosis and environmental exposures. &                   Mullen J &         J Occup Environ Med \\
            2000    &                                                                               Risk of cryptogenic fibrosing alveolitis in metal workers. &                  Hubbard R &                      Lancet \\
            2000    &  Occupational and environmental risk factors for idiopathic pulmonary fibrosis: a multicenter case-control study. Collaborating Centers. &             Baumgartner KB &              Am J Epidemiol \\
            2005    &                                                       Occupational and environmental factors and idiopathic pulmonary fibrosis in Japan. &                   Miyake Y &               Ann Occup Hyg \\
            2007    &                                                                                     Occupational exposure and severe pulmonary fibrosis. &                Gustafson T &                  Respir Med \\
            2008    &                                                     Occupational risks for idiopathic pulmonary fibrosis mortality in the United States. &                Pinheiro GA &  Int J Occup Environ Health \\
            2010    &                                            Risk factors for idiopathic pulmonary fibrosis in a Mexican population. A case-control study. &  García-Sancho Figueroa MC &                  Respir Med \\
            2011    &                                              Familial pulmonary fibrosis is the strongest risk factor for idiopathic pulmonary fibrosis. &            García-Sancho C &                  Respir Med \\
            2012    &                Occupational and environmental risk factors for idiopathic pulmonary fibrosis in Egypt: a multicenter case-control study. &                Awadalla NJ &     Int J Occup Environ Med \\
            2014    &    Effects of smoking, gender and occupational exposure on the risk of severe pulmonary fibrosis: a population-based case-control study. &                  Ekström M &                    BMJ Open \\
            2017    &                                      Occupational exposure and idiopathic pulmonary fibrosis: a multicentre case-control study in Korea. &                     Koo JW &       Int J Tuberc Lung Dis \\
    \bottomrule
    \end{tabular}
\caption{Previous IPF case-control studies looking at occupational exposure}
\label{table:papers}
\end{table}
                 
\section{Meta-analysis}
We found (insert tables + description)

\clearpage

%%%%%%%%%%%%%%%%%%%%%%%%%%
\makeatletter
 \def\@biblabel#1{#1}
\makeatother
%%%%%%%% gets rid of bracket around numbers in bibliography
%%%%%%%%%%%%%%%%%%%%%%%%%%%

\bibliographystyle{unsrtnat}
\bibliography{taskforce}


\end{document}
